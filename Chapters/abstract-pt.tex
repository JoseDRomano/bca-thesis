%!TEX root = ../template.tex
%%%%%%%%%%%%%%%%%%%%%%%%%%%%%%%%%%%%%%%%%%%%%%%%%%%%%%%%%%%%%%%%%%%%
%% abstract-pt.tex
%% NOVA thesis document file
%%
%% Abstract in Portuguese
%%%%%%%%%%%%%%%%%%%%%%%%%%%%%%%%%%%%%%%%%%%%%%%%%%%%%%%%%%%%%%%%%%%%

\typeout{NT FILE abstract-pt.tex}%

A classificação precisa dos subtipos de cancro da mama é essencial para
permitir estratégias de tratamento personalizadas e eficazes. No entanto, a
estratificação manual de pacientes com base apenas em critérios clínicos e
patológicos apresenta desafios significativos, devido à complexidade molecular
e heterogeneidade do cancro da mama. Neste contexto, a integração de
biomarcadores moleculares, tais como microRNAs (miRNAs), pequenos RNAs não
codificantes com papéis fundamentais na regulação pós-transcricional, oferece
uma via promissora para melhorar a precisão do diagnóstico e a possibilidade de
tratamentos personalizados.

Esta dissertação investiga como as técnicas de Machine Learning e Deep Learning
podem ser usadas para mapear perfis de expressão de miRNA para os subtipos
moleculares intrínsecos do cancro da mama: Luminal A, Luminal B,
HER2-enriquecido e Basal-like (Triplo-Negativo). A principal direção da
pesquisa está centrada na avaliação da eficácia da combinação de dados de miRNA
com variáveis clínicas para desenvolver classificadores de subtipos robustos e
interpretáveis. É dada especial atenção à comparação de abordagens
discriminatórias (por exemplo, regressão logística, XGBoost) com métodos
baseados em correlação ou espaço latente (por exemplo, DIABLO), a fim de
identificar as estratégias de modelação mais adequadas para a integração
multi-ómica e avaliar o impacto da incorporação de dados de diferentes naturezas.

Os resultados preliminares sugerem que os modelos de ML podem capturar com
sucesso as assinaturas de expressão de miRNA associadas a determinados
subtipos, embora o Luminal B e o HER2-enriched continuem a ser difíceis de
distinguir. Estas descobertas representam um primeiro passo para o objetivo
mais amplo deste trabalho: contribuir para o desenvolvimento de ferramentas de
apoio à decisão fiáveis e específicas para cada subtipo, que incorporem dados
moleculares e clínicos para uma estratificação mais precisa do cancro da mama e
um planeamento personalizado do tratamento.

% Palavras-chave do resumo em Português
% \begin{keywords}
% Palavra-chave 1, Palavra-chave 2, Palavra-chave 3, Palavra-chave 4
% \end{keywords}
\keywords{
  microRNAs \and
  cancro da mama \and
  biomarcadores \and
  aprendizagem automática aplicado à Medicina \and
  dados multi-ômics
}
