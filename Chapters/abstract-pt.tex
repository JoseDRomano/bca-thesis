%!TEX root = ../template.tex
%%%%%%%%%%%%%%%%%%%%%%%%%%%%%%%%%%%%%%%%%%%%%%%%%%%%%%%%%%%%%%%%%%%%
%% abstract-pt.tex
%% NOVA thesis document file
%%
%% Abstract in Portuguese
%%%%%%%%%%%%%%%%%%%%%%%%%%%%%%%%%%%%%%%%%%%%%%%%%%%%%%%%%%%%%%%%%%%%

\typeout{NT FILE abstract-pt.tex}%

A classificação precisa dos subtipos de cancro da mama é essencial para uma
abordagem clínica personalizada e eficaz. Nos últimos anos, os microRNAs
(miRNAs) emergiram como potentes biomarcadores devido ao seu papel na regulação
pós-transcricional e ao seu padrão de expressão específico em tecidos tumorais.
Esta dissertação explora a utilização de técnicas de Aprendizagem Automática e
Aprendizagem Profunda para mapear assinaturas de expressão de miRNAs a subtipos
moleculares de cancro da mama, nomeadamente: Luminal A, Luminal B,
HER2-enriched e Basal-like (ou Triple-Negative).

O trabalho desenvolvido inclui uma análise comparativa entre modelos
discriminativos (Logistic Regression, XGBoost) e métodos baseados em espaços
latentes (DIABLO, por exemplo). A pipeline experimental considera dados
multi-omnicos compostos por com 244 biópsias com 462 features (miRNAs e dados
clínicos), previamente pré-processadas e normalizadas.

Os resultados preliminares demonstram que é possível treinar modelos de
classificação que identifiquem padrões de expressão de miRNAs com resultados
robustos nas métricas de avaliação, salvo os subtipos HER2-enriched e Luminal-B
que aparentam ser mais difíceis de serparar. Estes achados sustentam a
viabilidade de integrar perfis de miRNAs e dados clínicos em sistemas de
suporte à decisão médica, e alinham-se com a visão desta dissertação:
contribuir para métodos mais precisos, interpretáveis e clinicamente relevantes
na estratificação do cancro da mama de modo a personalizar o tratmento de
acordo com o subtipo.

% Palavras-chave do resumo em Português
% \begin{keywords}
% Palavra-chave 1, Palavra-chave 2, Palavra-chave 3, Palavra-chave 4
% \end{keywords}
\keywords{
  microRNAs \and
  cancro da mama \and
  biomarcadores \and
  aprendizagem automática aplicado à Medicina \and
  dados multi-ômics
}
