%!TEX root = ../template.tex
%%%%%%%%%%%%%%%%%%%%%%%%%%%%%%%%%%%%%%%%%%%%%%%%%%%%%%%%%%%%%%%%%%%%
%% chapter2.tex
%% NOVA thesis document file
%%
%% Chapter with the template manual
%%%%%%%%%%%%%%%%%%%%%%%%%%%%%%%%%%%%%%%%%%%%%%%%%%%%%%%%%%%%%%%%%%%%

\typeout{NT FILE chapter2.tex}%

\chapter{Background and Related Work}

This chapter will provide the necessary background to understand the biological
context related to this work, as well as the work that has been already made in
the field of microRNA reaserch and its applications in breast cancer as a
biomarker and a subtype classifier. The chapter is divided into two main
sections: the first one will present the biological background, including the
central dogma of molecular biology, the role of \gls{mirna} in gene expression
regulation, and the characteristics of breast cancer and its subtypes. The
second section will review the related work in the field of microRNA research,
focusing on the use of \gls{mirna} as biomarkers and subtype classifiers in
breast cancer, as well as the challenges and limitations of current approaches.

\section{Biological Context}
The modern understanding of how genetic information is stored, interpreted, and
regulated in cells is based on a fundamentalprinciple known as the Central
Dogma of Molecular Biology. This concept, first formulated in
\textcites{discovery_dna_Watson1953The, updated_disc_of_dna_Pray2008DNA},
describes the unidirectional flow of genetic information in cells: from
\gls{dna} to \gls{rna} and from there to protein synthesis. According to this
model, genes encoded in \gls{dna} are transcribed into messenger \gls{rna} (or
mRNA), which in turn is translated into proteins—the functional molecules
responsible for most essential biological processes. This dogma has served as
the basis for much of the research in molecular biology and biotechnology.

However, in recent decades, it has become clear that this flow of information
is regulated in a much more complex way than initially thought. In particular,
it has been discovered that a substantial part of the genome is transcribed
into non-coding \gls{rna}, i.e., \gls{rna} that does not give rise to proteins
but plays fundamental regulatory roles. It is in this context that
\glossary{mirna} emerge, small \gls{rna} molecules with central functions in
the regulation of gene expression. Their discovery has broadened the classical
view of the central dogma, introducing new layers of post-transcriptional
control that decisively influence normal and pathological biological phenomena.

\subsection{DNA \& RNA - The Genetic Code}

At the molecular level, the genetic information of all living organisms is
encoded in a molecule called deoxyribonucleic acid (\gls{dna}). \gls{dna}
consists of two complementary strands arranged in a double helix structure,
with each strand consisting of a sequence of nucleotides. These nucleotides are
composed of a sugar-phosphate structure and one of four nitrogenous bases:
adenine (A), cytosine (C), guanine (G), and thymine (T)
\cite{ConceptsBiology_DNA}. When in the helix structure, these bases can only
be linked to their corresponding base: adenine can only be linked to thymine
and cytosine to guanine, and it is in the sequence of bases that the
instructions necessary for the synthesis of all the proteins that govern cell
structure and function are encoded.

\begin{figure}[h]
  \centering
  \includegraphics[width=0.6\textwidth]{dna.jpg}
  \caption{Structure of the \gls{dna} double helix}
  \label{fig:dna}
\end{figure}

The functional units of \gls{dna} are called genes, which are discrete
sequences that contain the instructions for producing proteins. However,
\gls{dna} itself cannot participate directly in protein synthesis. Instead, a
process called transcription is used to copy the information from a gene to a
\gls{rna} molecule as seen in \textcite{basic_biology_NCBI2002}. Unlike
\gls{dna}, \gls{rna} is single-stranded and uses uracil (U) instead of thymine
as one of its bases.

Among the various types of \gls{rna}, the best known is messenger \gls{rna}
(mRNA), which serves as an intermediary between genes and proteins. During
transcription, an mRNA molecule is synthesized as a complementary copy of a
gene, and this mRNA carries the genetic message from the \gls{dna} in the
nucleus to the ribosomes in the cytoplasm, where protein synthesis occurs. This
process, known as translation, is where the mRNA sequence is read in triplets
(called codons), each of which corresponds to a specific amino acid
\cite{central_dogma_molecular}.

%melhorar estrutura da imagem para ser como no paper
\begin{figure}[h]
  \centering
  \includegraphics[width=0.3\textwidth]{transcription1.png}
  \includegraphics[width=0.3\textwidth]{transcription2.png}
  \includegraphics[width=0.3\textwidth]{transcription3.png}
  \caption{Illustration of the transcription mechanism: (a) initiation, (b) elongation, and (c) termination}
  \label{fig:transcription}
\end{figure}

The set of rules by which the nucleotide sequence in messenger \gls{rna} is
translated into a sequence of amino acids is known as the genetic code. This
code is composed of triplets of nucleotides, called codons, where each codon
specifies one of the twenty standard amino acids used in protein synthesis seen
in \textcite{genetic_codeNovozhilov2008O} study.

The genetic code is described as redundant but unambiguous. Redundancy means
that most amino acids are encoded by more than one codon—for example, leucine
is specified by six different codons—which provides a certain degree of
robustness to the system. At the same time, the code is unambiguous because
each codon corresponds to only one amino acid; that is, a given codon does not
encode multiple amino acids \cite{ConceptsBiology_DNA}.

Another fundamental characteristic of the genetic code is its universality.
With very few exceptions, the same codons specify the same amino acids in
virtually all living organisms, from bacteria to humans. This evolutionary
conservation has been fundamental in enabling the development of many molecular
biology tools and biotechnological applications prooved by
\textcite{genetic_codeKoonin2017}.

Although the focus of molecular biology for decades has been on the coding
sequence of the genome—that is, the genes that give rise to proteins—it is now
known that a large part of the human genome is transcribed into \gls{rna} that
does not code for proteins. These non-coding \gls{rna} (ncRNA) molecules play
crucial regulatory roles in controlling gene expression. One of the most
studied groups within this class are \gls{mirna}, which appear to be central
elements in the fine-tuning of the genetic regulation process.

\subsection{MicroRNAs - The Regulators of Gene Expression}
\label{sec:microRNA}
\gls{mirna} are small non-coding \gls{rna} molecules, approximately $20$
to $25$ nucleotides in length, that play a key role in regulating gene
expression at the post-transcriptional level
\cite{regulatory_mecha_mirnaGulyaeva2016,
  first_mirna_Ambros1993,post_transcript_wightman1993}. Instead of encoding
proteins, \gls{mirna} act by controlling the production of proteins from genes.

\begin{figure}[h]
  \centering
  \includegraphics[width=0.8\textwidth]{mirna_process.png}
  \caption{The figure shows the process of gene expression: DNA is transcribed
    into mRNA, which is then translated into protein by the ribosome. \gls{mirna}
    are shown as regulators acting on the mRNA before translation.}
  \label{fig:mirna_mechanism}
\end{figure}

In simple terms, \textbf{\gls{mirna} function as molecular switches that bind
  to messenger \gls{rna} (mRNA) molecules}, blocking their translation into
protein or promoting their degradation. This mechanism depends on the degree of
complementarity between the \gls{mirna} sequence and that of the target mRNA:

\begin{itemize}
  \item When there is high complementarity, the mRNA tends to be degraded;
  \item When complementarity is partial, the \gls{mirna} generally acts by inhibiting
        translation without destroying the mRNA.
\end{itemize}

A study made by \textcite{role_mirna_Calaf2023} demonstrates the high
effienciency of this mechanism of regulation: a \textbf{single \gls{mirna} can
  control dozens to hundreds of different genes}, and it is estimated that more
than $60\%$ of human coding genes are targeted for regulation by \gls{mirna}.

Due to this broad regulatory capacity, \gls{mirna} play a central role in
multiple cellular processes such as proliferation, differentiation, apoptosis,
and stress response. Consequently, changes in \gls{mirna} expression profiles
are associated with several diseases, including cancer, neurodegenerative and
cardiovascular diseases. In an oncological context, \gls{mirna} can act as
oncogenes (promoting tumor growth) or as tumor suppressors, depending on the
biological context and cell type as shown by
\textcite{regulatory_mecha_mirnaGulyaeva2016}.

Due to their specificity, stability, and direct involvement in relevant
molecular mechanisms, \gls{mirna} have been extensively investigated as
promising biomarkers for diagnosis, prognosis, and subtype stratification in
various diseases—including cancer.

\subsection{Cancer - A Complex Disease}
Cancer is a disease characterized by the uncontrolled proliferation of
transformed cells, which can invade neighboring tissues and spread to other
parts of the body through processes such as metastasis. This definition, based
on \textcite{NCI2021,def_of_cancer_Brown2023}, has recently been expanded to
recognize the role of natural selection in the evolution of cancer: it is a
cellular system that continuously evolves, adapting to internal and external
pressures to ensure its survival.

Under normal conditions, the body's cells divide only when necessary, die when
damaged or obsolete, and are replaced by new ones. However, in cancer, this
biological balance is disrupted: abnormal cells gain the ability to multiply
independently of the body's signals and to resist programmed cell death
(apoptosis). These transformed cells become autonomous units that not only
ignore normal growth controls but also interact with the tumor microenvironment
to promote their own survival, using angiogenesis, immune evasion, and other
adaptive mechanisms \cite{def_of_cancer_Brown2023,NCI2021}.

The result is a heterogeneous cell population, subject to natural selection
within the human body. Cells that acquire adaptive advantages (e.g., higher
proliferation rate, drug resistance, or migration ability) tend to prevail,
making cancer a constantly evolving disease \cite{def_of_cancer_Brown2023}.

Although cancer can arise in virtually any tissue, not all cellular changes are
malignant. There are precancerous conditions, such as \textit{hyperplasia or
  dysplasia}, which represent an increase in the number of cells or changes in
their morphology, but which do not yet invade surrounding tissues.

Progression of cancer is a complex process that involves the
\textbf{acquisition of invasive and metastatic capacity—properties that
  distinguish malignant tumors from benign ones}. This process can be silent for
years, until more severe symptoms arise, often related to the invasion of vital
organs.

\subsection{Breast Cancer \& its Subtypes}

Breast cancer is the most commonly diagnosed cancer in women worldwide and is
one of the leading causes of cancer death in developed and developing countries
as seen in \textcite{BreastEpidemiology_Romanowicz2022,
  updatedbca_Hong2022Breast}. It is estimated that \textbf{one in eight women}
will be diagnosed with this disease during their lifetime, although it can also
affect men—albeit with a much lower incidence.

Most breast tumors are originated in the epithelial cells of the ducts or
lobules of the breast, which acquire malignant properties after the
accumulation of genetic and epigenetic changes. These events alter the normal
control of cell proliferation, differentiation, and apoptosis, allowing for
unregulated tumor growth \cite{origins_and_evolution_bca_Polyak2007}.

\begin{figure}
  \centering
  \includegraphics[width=0.8\textwidth]{bca_anatomy.png}
  \caption{In (A) we have a normal breast tissue, while in (B) we can see
    the presence of a malignant tumor \cite{bca_anatomy_figure_Muthu2020}.}
  \label{fig:breast_cancer_anatomy}
\end{figure}

The development of the disease is associated with a set of well-established
risk factors, which include:
\begin{itemize}
  \item Age and family history of the disease;
  \item Hereditary genetic mutations, especially in the \textit{BRCA1} and
        \textit{BRCA2} genes;
  \item Prolonged exposure to endogenous or exogenous hormones (e.g., early menarche,
        late menopause, hormone therapy);
  \item Environmental and behavioral factors, such as obesity, physical inactivity,
        alcohol consumption, and a diet rich in saturated fats
        \cite{BreastEpidemiology_Romanowicz2022,clinical_implication_bca_Adamo2015}.
\end{itemize}

From a molecular and clinical point of view, \textbf{breast cancer is highly
  heterogeneous}. Each tumor may have unique combinations of genetic alterations,
signaling pathways, and gene expression profiles, which are reflected in
different clinical behaviors, degrees of aggressiveness, and response to
treatment
\cite{origins_and_evolution_bca_Polyak2007,diff_bca_usa_Howlader2018}.

Early detection is crucial for prognosis. When diagnosed in its early stages,
breast cancer has survival rates of over $90\%$. However, in more advanced
stages, especially when metastases appear, controlling the disease becomes
substantially more difficult and the therapeutic goal shifts from curative to
palliative \cite{updatedbca_Hong2022Breast,clinical_implication_bca_Adamo2015}.

The therapeutic approach is typically multimodal, combining surgery,
radiotherapy, chemotherapy, hormone therapy, and targeted or biological
therapies, depending on the characteristics of the tumor and the patient's
general condition. The most significant advance in the last decade has been the
transition from a uniform model to a personalized treatment approach, tailored
to the molecular subtype and individual risk as studied by
\textcite{BreastEpidemiology_Romanowicz2022}.

In addition, a study made by \textcite{origins_and_evolution_bca_Polyak2007}
recognized that breast tumors are not static entities. Due to phenomena of
intra-tumor heterogeneity and clonal evolution, tumors adapt to the selective
pressure of treatments, often leading to the development of therapeutic
resistance and disease progression.

Given the molecular complexity and clinical diversity of breast tumors, it was
established in the papers
\textcite{clinical_implication_bca_Adamo2015,bc_subtypes_Prat2015Clinical} that
breast cancer is not a single disease but rather a collection of biologically
distinct entities that arise from a common anatomical site. This heterogeneity
is reflected in major differences in tumor progression, metastatic behavior,
response to therapy, and long-term prognosis.

To better capture this complexity and inform clinical decision-making,
researchers from various studies, such as
\textcite{bc_molecular_Perou2000,bc_subtypes_Prat2015Clinical}, have developed
a molecular classification system that subdivides breast tumors into intrinsic
subtypes. These subtypes are defined based on the expression status of three
key biomarkers: \textbf{estrogen receptor (ER)}, \textbf{progesterone receptor
  (PR)}, and \textbf{human epidermal growth factor receptor 2 (HER2)} as well as
other proliferation indices (e.g., Ki-67) and gene expression patterns. This
classification underpins modern precision oncology approaches and has profound
implications for therapy and prognosis.

As shown in the table \ref{tab:bc_subtypes_summary}, breast cancer can be
classified into four main intrinsic subtypes based on the expression of these
biomarkers and other molecular characteristics:

%%% REMOVER O h para !t no final1!!!!!!asdonasodniasoindaosindoi
\renewcommand{\arraystretch}{1.3}
\begin{table}[h]
  \centering
  \small
  \caption{Summary of intrinsic breast cancer subtypes and typical characteristics of each one.\newline\textit{References:} \cite{clinical_implication_bca_Adamo2015}, \cite{diff_bca_usa_Howlader2018}, \cite{bc_subtypes_Prat2015Clinical}, \cite{updatedbca_Hong2022Breast}, \cite{tnbc_therapies_Mahalingam2020The}.}
  \label{tab:bc_subtypes_summary}
  \begin{tabularx}{\textwidth}{l l l l X}
    \toprule
    \textbf{Subtype}  & \textbf{Receptors / HER2} & \textbf{Prolif.} & \textbf{Prognosis} & \textbf{Treatment}          \\
    \midrule
    Luminal A         & ER+/PR+, HER2−            & Low              & Favorable          & Endocrine only              \\
    \midrule
    Luminal B         & ER+, PR↓, HER2±           & High             & Intermediate       & Hormone ± Chemo ± Anti-HER2 \\
    \midrule
    HER2-enriched     & HER2+, ER−, PR−           & High             & Improved           & Anti-HER2 + Chemo           \\
    \midrule
    Basal-like / TNBC & ER−, PR−, HER2−           & High             & Poor               & Chemo; ± PARP/IO (selected) \\

    \bottomrule
  \end{tabularx}

  \vspace{1ex}
  \raggedright
  \footnotesize
  \textit{Acronyms:} Chemo = Chemotherapy , IO = Immunotherapy .
\end{table}

Recent evidence by
\textcite{intratumor_heterogeneity_Yeo2017,origins_and_evolution_bca_Polyak2007}
suggests that multiple subtypes can coexist within the same tumor (a phenomenon
called intra-tumor heterogeneity). This complexity contributes to therapeutic
resistance and disease progression.

\subsection{Nucleic acids as gene therapies}
\label{sec:nucleic_acids_gene_therapies}
% Now that we have a better understanding of limitations that most of cancer treatments 
% have, \gls{mirna} have emerged as a promising tool in the field of cancer
% research and treatment. 

% Considering them as the "natural regulators" of the human body, we can leverage 
% that regulating mechanism and use it to 

\newpage
\section{Related work}

This section will present a critical review of computational approaches
developed to date to explore the potential of \gls{mirna} as biomarkers in the
context of oncology, covering both \gls{bc} and other malignant neoplasms. The
contributions of \gls{ml} and \gls{dl} models applied to the task of
classifying different cancer subtypes will also be analyzed, with a special
focus on methodologies that integrate molecular data with data from other
nature, like clinical characteristics for example.

\gls{ml}, a branch of \gls{ai}, involves
developing computational models that learn from data to make predictions or
decisions. These models are typically trained using either supervised
learning, where the target outcomes are known and used during training, or
unsupervised learning, in which no explicit labels or outcome variables are
provided. In both paradigms, the goal is to uncover meaningful patterns in the
data that can be used to generate predictive insights, such as detecting the
presence of cancer, estimating survival probabilities, or stratifying patients
into risk categories. \gls{ml} techniques are particularly valuable when dealing with
unstructured or complex clinical datasets, as is often the case in oncology.

In recent years, the application of \gls{ml} algorithms to the field of
biomedicine has led to significant advances in the analysis of complex and
high-dimensional data, including the expression of \gls{mirna} in cancer
\cite{role_of_ai_giger2021}. In this context, several studies have explored the
use of computational models for the classification of tumor subtypes and/or the
identification of discriminative biomarkers, with promising results but also
with important limitations.

In this context, we will review and analyze scientific works that have
leveraged \gls{ml} algorithms in contexts similar to the stratification of
\gls{bc} subtypes based on \gls{mirna} expression profiles, complementing them
with data of other types (multi-omics data). For each study, it will be
important to define the specific work in question so that we can analyze the
methodology, algorithms used, and results obtained, all based on the specific
context of the research in question, in order to capture and consolidate a
ground on which we can work. At the end of the review, we will discuss how
these contributions inform and substantiate the methodological choices made in
the present work, justifying, whenever possible, the algorithmic and
experimental choices based on the available scientific evidence.

\subsection{Leveraging \gls{ai} models for cancer classification}
% HERE: começar com cancer classification em geral -> depois falar de 
% Papers a usar: 4, 10 , 11, 12 ( estes dois últimos são com imagens, mas podemos abordar)

The classification of different types of cancer using computational models has
been one of the most explored areas within the application of \gls{ai} to
medicine. Let's take a look at the work of
\textcite{ai_in_dermacancer_esteva2017}, a remarkable advance in this “new”
relationship between computers and dermatology, where deep neural networks
(Figure \ref{fig:DNN}) have demonstrated capabilities comparable to those of
human experts in the diagnosis of malignant skin lesions.

\begin{figure}[htbp]
  \centering
  \includegraphics[width=0.7\textwidth]{/Users/JoseRomano/Documents/Tese/bca-thesis/Chapters/Figures/NN.png}
  \caption{Structure of an deep neural network (DNNs). It shows the input, hidden, and output layers, with connections between neurons responsible for processing information \cite{analyticsvidhya_image}.}
  \label{fig:DNN}
\end{figure}

This work was made possible by the use of an architecture based on
convolutional neural networks (CNNs) - a type of DNN that is particularly
effective in image processing (Figure \ref{fig:CNN_derma}). CNNs work by
applying convolutional filters that extract visual patterns at different levels
of complexity, allowing the model to identify relevant features directly from
the image pixels, without the need for specialized preprocessing
\cite{CNN_Albawi2017}.

\begin{figure}[htbp]
  \centering
  \includegraphics[width=1.0\textwidth]{/Users/JoseRomano/Documents/Tese/bca-thesis/Chapters/Figures/CNN_derma.png}
  \caption{Schematic of the CNN used by \textcite{ai_in_dermacancer_esteva2017} with Inception v3 architecture, adapted to classify skin lesions based on clinical images. The network generates a probability distribution over clinical classes, based on a structured medical taxonomy.}
  \label{fig:CNN_derma}
\end{figure}

The biggest problem with this methodology is that these architectures require a
large number of cases (positive and negative) in order to learn the necessary
patterns. In this case, 129{,}450 clinical images covering more than 2{,}000
different diseases were used. Some factors that determined the good results of
this model were:

\begin{enumerate}
  \item The photographic variability of the samples on which it was trained, since they
        covered not only images taken with mobile phones, but also dermoscopy images;

  \item The manipulation of images during training, enlarging and inverting them to
        increase the adaptability and robustness of the model;

  \item The use of a structured medical taxonomy (Figure \ref{fig:taxonomy}), built on
        clinical and visual criteria, which allowed for the organization of more than
        2{,}000 diseases into a hierarchy of 757 fine-grained training classes, such as
        \textit{acrolentiginous melanoma} and \textit{amelanotic melanoma}.
\end{enumerate}

\begin{figure}[htbp]
  \centering
  \includegraphics[width=0.5\textwidth]{/Users/JoseRomano/Documents/Tese/bca-thesis/Chapters/Figures/taxonomy.png}
  \caption{A subset of the hierarchical taxonomy developed in the study by \textcite{ai_in_dermacancer_esteva2017}, with diseases organized by clinical and visual similarity into three major groups: benign, malignant, and non-neoplastic.}
  \label{fig:taxonomy}
\end{figure}

The result? A computational model that not only achieved performance comparable
to that of certified dermatologists, but in several scenarios even demonstrated
superiority over average human performance, verifiably by this confusion matrix
(Figure \ref{fig:conf-matrix-docs}). The trained convolutional neural network
was able to classify two critical clinical cases with high accuracy:
keratinocytic carcinomas versus benign seborrheic keratoses, and malignant
melanomas versus benign nevi. In these binary scenarios, it obtained areas
under the curve (\textit{AUC}) of 0.96 and 0.94, respectively (Figure
\ref{fig:AUC_dnn_model}) - values higher than those obtained by dermatologists
in the same tasks. \textit{AUC} is a metric that quantifies a model's ability
to distinguish between classes, with values close to 1 indicating excellent
performance.

\begin{figure}[htbp]
  \centering
  \begin{subfigure}[b]{0.45\textwidth}
    \centering
    \includegraphics[width=\textwidth]{/Users/JoseRomano/Documents/Tese/bca-thesis/Chapters/Figures/conf-matrix-docs.png}
    %\caption{(a)}
    \label{fig:conf-matrix-docs}
  \end{subfigure}
  \hfill
  \begin{subfigure}[b]{0.45\textwidth}
    \centering
    \includegraphics[width=\textwidth]{/Users/JoseRomano/Documents/Tese/bca-thesis/Chapters/Figures/AUC_derma_2.png}
    %\caption{(b)}
    \label{fig:AUC_dnn_model}
  \end{subfigure}
  \caption{Performance evaluation of the convolutional neural network (CNN) in skin lesion classification. (a) Confusion matrices of CNN and two dermatologists. The concentration on the diagonal indicates correct classifications; CNN shows less dispersion and better overall performance \cite{ai_in_dermacancer_esteva2017}. (b) Reliability of the CNN demonstrated by \textit{AUC} curves on a larger, independent dataset \cite{ai_in_dermacancer_esteva2017}.}
  \label{fig:AUC_derma_total}
\end{figure}

Furthermore, in more complex scenarios with multiple classes (three and nine
disease categories), the model maintained \textbf{remarkable levels of
  accuracy} (72.1\% and 55.4\%), surpassing or equaling human experts. The
robustness of the methodology, that is, its ability to maintain performance
under different conditions or test data, was also confirmed in larger test
sets, where the network's performance remained stable, with \textbf{minimal
  variations in evaluation metrics}. From a technical standpoint, it was an
efficient, scalable system with \textbf{potential for application in mobile
  devices}, which gives it relevant clinical applicability, especially in
contexts with limited access to specialists.

Internal analyses further reinforced confidence in the model, showing that it
learned consistent clinical representations: the network tended to
\textbf{group diseases with similar visual characteristics} (Figure
\ref{fig:tsne_derma}) and focused its attention on the damaged areas of the
images, ignoring irrelevant regions such as background or healthy skin -
promising evidence of \textit{automated clinical focus} with real practical
utility.

\begin{figure} [h]
  \centering
  \includegraphics[width=0.5\textwidth]{/Users/JoseRomano/Documents/Tese/bca-thesis/Chapters/Figures/t_sne_derma.png}
  \caption{t-SNE projection of the internal representations of the last hidden layer of the CNN \cite{ai_in_dermacancer_esteva2017}. The different classes of lesions are grouped into distinct clouds, revealing the model's ability to extract relevant discriminative features.}
  \label{fig:tsne_derma}
\end{figure}

\noindent\rule{\linewidth}{0.4pt}

The effectiveness demonstrated in the previous project shows the magnitude of
the benefits that \gls{ai} can bring to the world of medicine, helping doctors
diagnose and stratify diseases with an accuracy that, in some cases, surpasses
that of human specialists. This capability is not limited to imaging data: it
also extends to the field of molecular data and dermatology, as demonstrated by
the study by \textcite{bca_subtypes_with_ml_Wu_2021}, which applies machine
learning algorithms to the task of classifying breast cancer subtypes (in this
case, the goal was to distinguish between Triple Negative, or basal-like, and
non-Triple Negative tumors, since TNBC is the most deadly cancer with the most
difficult prognosis, as we saw in the table).

In the study by \textcite{bca_subtypes_with_ml_Wu_2021}, when working with gene
expression data from thousands of patients, additional challenges arise related
to the high dimensionality of the data, requiring robust feature selection
methods and predictive models capable of dealing with complex and often
non-linear correlations. This type of study brings us closer not only to the
context of this thesis, but also to the type of challenges we will encounter
and how we can take advantage of the methodologies used in this paper to
achieve good results. From all the algorithms that were tested, Support Vector
Machine (\textit{SVM}) stood out for its performance. This approach allowed the
authors to achieve high levels of accuracy, sensitivity, and specificity,
demonstrating the potential of \gls{ai} in classifying breast cancer subtypes
based on genomic information. The type of information for this study was the
RNA-Sequence (RNA-Seq) profiles made available by The Cancer Genome Atlas
(TCGA), a public database containing thousands of tumor samples characterized
at the genomic level. This dataset, after pre processing, had 934 tumor samples
and over 57{,}000 genes per sample, a typical high-dimensionality scenario
where the number of variables far exceeds the number of observations, and
considering that not all of them are necessary or have a great impact,
\textbf{differential expression analysis was applied} - a bioinformatics
technique used to detect which genes are significantly more or less expressed
between different conditions - resulting in the selection of 5{,}502
differentially expressed genes, which served as input for the predictive models
(a large reduction of over 50{,}000 genes). This step corresponds to feature
selection, which is essential in problems where there is a high risk of
\textit{overfitting} - that is, when the model memorizes the training data but
fails to generalize to new examples.

\begin{figure} [h]
  \centering
  \includegraphics[width=1.0\textwidth]{/Users/JoseRomano/Documents/Tese/bca-thesis/Chapters/Figures/overfit.png}
  \caption{Examples of underfitting, proper fitting, and overfitting.
    From left to right: the model underfits the data, fits it appropriately, and overfits by capturing noise instead of the underlying pattern \cite{overfiting_ailab_mti_image}.}
  \label{fig:overfitting}
\end{figure}

Now that the dataset had been reduced to a more informative and manageable
subset of features, the authors moved on to the predictive modeling phase. This
is a crucial moment in the \gls{ml} pipeline, where the ability of different
algorithms to learn discriminative patterns present in the data is tested—in
this case, distinguishing between TNBC and non-TNBC tumors based on the
expression levels of selected genes. Several classic supervised learning
algorithms were then evaluated, representing different approaches to the
classification task:

\begin{itemize}
  \item \textbf{K-nearest Neighbors (\textit{kNN})}: classifies new data based on the K nearest neighbors in the feature space \cite{knn_article}.
        \begin{figure} [h]
          \centering
          \includegraphics[width=0.45\linewidth]{/Users/JoseRomano/Documents/Tese/bca-thesis/Chapters/Figures/knn-example.jpg}
          \caption{Illustration of the k-Nearest Neighbors (k-NN) classification process. The top-left panel shows the initial labeled data (Class A in yellow, Class B in purple) and a new unlabeled sample (?). The top-right panel demonstrates the calculation of distances from the new sample to all existing points. The bottom panel shows the selection of the k=3 nearest neighbors and class assignment based on majority voting, resulting in the classification of the new sample.}
        \end{figure}

  \item \textbf{Naïve Bayes (\textit{NB})}: uses Bayes' Theorem to estimate the most likely class of a sample, assuming that the input variables are independent of each other \cite{naivebayes_Watson2001}.
        \[
          P(C_k \mid \mathbf{x}) = \frac{P(C_k) \prod_{i=1}^n P(x_i \mid C_k)}{P(\mathbf{x})}
        \]

  \item \textbf{Decision Tree (\textit{DT})}: a model that makes decisions through a hierarchical tree-shaped structure, where each internal node represents a condition on a variable, and each branch represents a possible outcome of that condition. The process continues until it reaches a leaf node, which indicates the final class or value \cite{decision_trees_Jijo2021}.
        \begin{figure} [h]
          \centering
          \includegraphics[width=0.45\linewidth]{/Users/JoseRomano/Documents/Tese/bca-thesis/Chapters/Figures/dt-example.png}
          \caption{Example of a Decision Tree for Classification Based on Attributes: Income, Age, Student Status, and Credit Rating (CR). The tree predicts a binary decision outcome (Yes/No) using hierarchical decision rules \cite{decision_trees_Jijo2021Classification}.}
        \end{figure}

  \item \textbf{Support Vector Machine (\textit{SVM})}: This algorithm constructs an optimal hyperplane that best separates samples from different classes. The goal of SVM is to maximize the margin between the two classes for better generalization. Support vectors are the data points that lie closest to the hyperplane and define its position.
        \begin{figure} [h]
          \centering
          \includegraphics[width=0.45\linewidth]{/Users/JoseRomano/Documents/Tese/bca-thesis/Chapters/Figures/SVM.png}
          \caption{Visualization of a Support Vector Machine (SVM) classifier. The red line represents the optimal hyperplane that separates two classes (blue and green points), while the dashed lines indicate the margins.}
        \end{figure}
\end{itemize}

In order to evaluate a model, performance metrics are used to assess how well
it performs across different dimensions, such as accuracy, relevance, and
sensitivity to different classes. These metrics provide quantitative insight
into the strengths and limitations of a classification algorithm, allowing
researchers and practitioners to make informed decisions when comparing models
or tuning parameters. Evaluating models using multiple metrics is especially
important in scenarios involving imbalanced datasets, where a single metric
(such as accuracy) may not provide a complete picture. The most commonly used
evaluation metrics in classification tasks, according to
\textcite{metrics_models_yaseen2021}, are:

\begin{enumerate}
  \item \textbf{Accuracy} \\
        Accuracy is the ratio of correctly predicted instances to the total number of instances. It measures the overall effectiveness of a classification model.
        \[
          \text{Accuracy} = \frac{TP + TN}{TP + TN + FP + FN}
        \]
        where:
        \begin{itemize}
          \item TP = True Positives
          \item TN = True Negatives
          \item FP = False Positives
          \item FN = False Negatives
        \end{itemize}

  \item \textbf{Precision} \\
        Precision measures the proportion of correctly predicted positive observations to the total predicted positive observations. It reflects the model’s ability to return only relevant results.
        \[
          \text{Precision} = \frac{TP}{TP + FP}
        \]

  \item \textbf{Recall (Sensitivity or True Positive Rate)} \\
        Recall is the ratio of correctly predicted positive observations to all actual positives. It indicates the model’s ability to identify all relevant cases.
        \[
          \text{Recall} = \frac{TP}{TP + FN}
        \]

  \item \textbf{F1-Score} \\
        The F1-Score is the harmonic mean of Precision and Recall, providing a balance between the two. It is particularly useful when the class distribution is imbalanced.
        \[
          \text{F1-Score} = 2 \times \frac{\text{Precision} \times \text{Recall}}{\text{Precision} + \text{Recall}}
        \]

  \item \textbf{Support} \\
        Support refers to the number of actual occurrences of each class in the dataset. While not a performance metric per se, it helps in understanding how many examples a classifier is making predictions on for each class.
\end{enumerate}

Now that we are familiar with the main concepts of classification models and
the metrics used to evaluate them, let us return to the study by Wu and Hicks
(2021) in light of these indicators. The analysis of the results, presented in
the following table, allows us to see more clearly how each algorithm performed
in the task of distinguishing between TNBC and non-TNBC, highlighting the
performance of SVM in virtually all scenarios evaluated.

\begin{table}[h!]
  \centering
  \begin{tabular}{|l|c|c|c|c|c|}
    \hline
    \textbf{Model} & \textbf{Accuracy} & \textbf{Recall} & \textbf{Precision} & \textbf{F1-score} & \textbf{Specificity} \\
    \hline
    kNN            & 0.81              & 0.76            & 0.80               & 0.78              & 0.86                 \\
    Naïve Bayes    & 0.84              & 0.81            & 0.83               & 0.82              & 0.88                 \\
    Decision Tree  & 0.88              & 0.86            & 0.87               & 0.86              & 0.89                 \\
    \textbf{SVM}   & \textbf{0.90}     & \textbf{0.87}   & \textbf{0.90}      & \textbf{0.88}     & \textbf{0.90}        \\
    \hline
  \end{tabular}
  \caption{Performance of models in classifying breast cancer subtypes (TNBC vs non-TNBC) based on differential gene expression.}
  \label{tab:model-performance}
\end{table}

In the complete gene set, SVM achieved 90\% accuracy, 87\% recall, and 90\%
specificity—metrics that indicate, respectively, the proportion of correct
classifications, the ability to correctly identify TNBC cases, and the ability
to avoid false positives. The analysis of these metrics is essential to
correctly interpret the results in a clinical context, where the consequences
of classification errors can be significant.

To validate the robustness of the differentially expressed gene selection
approach, the authors compared it with other classic feature selection methods,
such as \textit{SVM-RFE} (an iterative technique that removes the least
relevant features based on \textit{SVM} weights), \textit{Relief} (which
weights features based on their correlation with the class), \textit{ARCO}, and
\textit{mRMR} (which maximizes relevance and minimizes redundancy between
variables). Even so, the model based on differential expression and
\textit{SVM} demonstrated better performance in most cases, confirming the
soundness of the strategy adopted.

The study further deepened the analysis of feature importance by evaluating the
performance of models with different sizes of gene subsets, from the initial
5,502 to only 16 genes. Interestingly, \textit{SVM} performance remained high
even with reduced sets, achieving the best results with 256 genes. This
stability suggests that discriminative information is concentrated in a small
subset of features, which is relevant for scenarios with a low number of
samples and a large number of features, as we have in this dissertation, and
quite positive because:

\begin{itemize}
  \item it allows molecular tests to be cheaper and faster since fewer genes are
        needed;
  \item it makes models easier to validate in a clinical setting since it is simpler to
        obtain quality samples with few targets;
  \item greater interpretability for physicians.
\end{itemize}

The analysis of the two studies presented here allows us to consolidate a
fundamental idea for this dissertation: \gls{ml} and \gls{dl} algorithms
demonstrate a remarkable ability to \textbf{classify different types of cancer
  based on complex data}, whether imaging or molecular. From deep neural networks
applied to dermatological imaging to discriminative algorithms used to analyze
gene expression, these models have proven to be effective tools for supporting
clinical decisions, sometimes proving superior to human experts. More than just
efficient classifiers, these systems have also proven to be interpretable,
robust, and applicable in real clinical scenarios. Both the image-based
approach \cite{ai_in_dermacancer_esteva2017} and the gene expression-based
approach \cite{bca_subtypes_with_ml_Wu_2021} have faced and overcome challenges
typical of medical practice and biomedical research: sample scarcity, high
dimensionality, and the need for models with good overall performance, but also
\textbf{confidence in the prediction of clinically critical cases}.

These findings provide the \textbf{conceptual support needed to explore a more
  specific direction}: the use of \gls{ml} models to discover and validate
\textbf{\gls{mirna} as biomarkers} in oncology. Like coding genes, \gls{mirna}
carry rich and discriminative information about the biological state of cells
and have shown promise in the stratification of tumor subtypes. The next
section addresses precisely this line of research, focusing on how ML has been
used to reveal \gls{mirna} signatures with diagnostic and prognostic value - a
foundation point for the objectives of this dissertation.

\subsection{\gls{ml} unravelling \gls{mirna} as biomarkers}
% Papers a usar: 1, 2 e 3

As previously discussed, artificial intelligence models have demonstrated
proficiency in the task of classificating tumor type, or subtype, by either
analysing clinical images, high-dimensional genomic data and other forms of
data. In this thesis, our focus is into a more specific and pertinent domain:
the identification of \gls{mirna} as biomarkers in oncological contexts through
\gls{ml} techniques. We know for a fact that \gls{mirna} are a class of small
molecules that have been demonstrated to possess a substantial regulatory
capacity over gene expression \ref{sec:microRNA}, but even though this capacity
distinguishes them as optimal candidates for utilization as molecular
biomarkers, the number of \glossary{mirna} expressed in human tissues, in
conjunction with their variability across individuals, demands the
implementation of robust computational methodologies.

In this context, \gls{ml} models have gained prominence as powerful tools for
revealing latent patterns in \gls{mirna} expression data, allowing the
identification of subsets with diagnostic, prognostic, or tumor subtype
stratification value. This research trajectory is particularly auspicious, as
it proffers more readily implementable, non-invasive methodologies that can be
substantiated within a clinical environment. That's what we are going to
explore in this section, where we we will present three papers that illustrate
different stages of this scientific effort: the first being a general reference
of the type of work that we will be analyzing giving us a glampse of how
\gls{ml} can work in this context; the second one being a more detailed
\gls{ml} pipeline applied to the identification of \gls{mirna} as biomarkers
for gastric cancer, and lastly, the intersection of this kind of approach with
\glossary{bc}, which is the central carcinoma focus of this thesis.

\noindent\rule{\linewidth}{0.4pt}



\subsection{Comparison with thesis approach}
% Papers a usar: 2, 4 e 10

\subsection{Other relevant work}
% HERE: applying ml in diagnosis and prognosis + ai in oncological nanomedicine
% Papers a usar: 6,8,5,7

\subsection{Conclusion}