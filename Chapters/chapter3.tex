%!TEX root = ../template.tex
%%%%%%%%%%%%%%%%%%%%%%%%%%%%%%%%%%%%%%%%%%%%%%%%%%%%%%%%%%%%%%%%%%%%
%% chapter3.tex
%% NOVA thesis document file
%%
%% Chapter with a short latex tutorial and examples
%%%%%%%%%%%%%%%%%%%%%%%%%%%%%%%%%%%%%%%%%%%%%%%%%%%%%%%%%%%%%%%%%%%%

\typeout{NT FILE chapter3.tex}%
\chapter{Preliminary Work}

This chapter presents the initial experiments using machine learning for breast
cancer molecular subtype classification based on the patients’ microRNAs
expression levels. Early work focused on testing baseline \gls{ml} models
(Logistic Regression, SVM, Random Forest) to assess their classification
performance using datasets such as TCGA-BRCA. These initial experiments laid
the foundation for the extensive work that followed and ultimately led to the
publication of a scientific paper [\ref{ann:EPIA_paper}] at the EPIA 2025
Conference under the category of "AI in Medicine" by our thesis group. The
experiments will be divided into two main sections: the first covers the steps
using an early dataset version, which raised concerns due to unclear
normalization procedures — potentially leading to data leakage. In the second
section, the same dataset is used, but I re-processed it from scratch to ensure
a clean and leakage-free setup.

\section{Possible Data Leakage Dataset}

The first dataset used was the TCGA-BRCA dataset, which contains microRNA
expression levels for breast cancer patients. The dataset was obtained from the
TCGA database and pre-processed. The table asda portraits the number of samples
and features in the dataset (including some clinical features), as well as the
distribution of molecular subtypes in the graphic
\ref{fig:subtype_distribution}.

