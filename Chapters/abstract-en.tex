%!TEX root = ../template.tex
%%%%%%%%%%%%%%%%%%%%%%%%%%%%%%%%%%%%%%%%%%%%%%%%%%%%%%%%%%%%%%%%%%%%
%% abstract-en.tex
%% NOVA thesis document file
%%
%% Abstract in English
%%%%%%%%%%%%%%%%%%%%%%%%%%%%%%%%%%%%%%%%%%%%%%%%%%%%%%%%%%%%%%%%%%%%

\typeout{NT FILE abstract-en.tex}%

Accurate classification of breast cancer subtypes is essential for a
personalized and effective clinical approach. In recent years, microRNAs
(miRNAs) have emerged as potent biomarkers due to their role in
post-transcriptional regulation and their specific expression pattern in tumor
tissues. This dissertation explores the use of Machine Learning and Deep
Learning techniques to map miRNA expression signatures to molecular subtypes of
breast cancer, namely: Luminal A, Luminal B, HER2-enriched, and Basal-like (or
Triple-Negative).

The work developed includes a comparative analysis between discriminative
models (Logistic Regression, XGBoost) and methods based on latent spaces
(DIABLO, for example). The experimental pipeline considers multi-omnics data
composed by 244 patient data with 462 features (miRNAs and clinical data),
previously pre-processed and normalized.

Preliminary results demonstrate that it is possible to train classification
models that identify miRNA expression patterns with robust results in the
evaluation metrics, except for the HER2-enriched and Luminal-B subtypes, which
appear to be more difficult to separate. These findings support the feasibility
of integrating miRNA profiles and clinical data into medical decision support
systems and are in line with the vision of this dissertation: to contribute to
more accurate, interpretable, and clinically relevant methods for breast cancer
stratification in order to personalize treatment according to subtype.

% Palavras-chave do resumo em Inglês
% \begin{keywords}
% Keyword 1, Keyword 2, Keyword 3, Keyword 4, Keyword 5, Keyword 6, Keyword 7, Keyword 8, Keyword 9
% \end{keywords}
\keywords{
  microRNAs \and
  breast cancer \and
  biomarkers \and
  machine learning applied to Medicine \and
  multi-omics data
}
