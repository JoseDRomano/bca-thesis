%!TEX root = ../template.tex
%%%%%%%%%%%%%%%%%%%%%%%%%%%%%%%%%%%%%%%%%%%%%%%%%%%%%%%%%%%%%%%%%%%%
%% abstract-en.tex
%% NOVA thesis document file
%%
%% Abstract in English
%%%%%%%%%%%%%%%%%%%%%%%%%%%%%%%%%%%%%%%%%%%%%%%%%%%%%%%%%%%%%%%%%%%%

\typeout{NT FILE abstract-en.tex}%

Accurate classification of breast cancer subtypes is essential for enabling
personalized and effective treatment strategies. However, the manual
stratification of patients based solely on clinical and pathological criteria
presents significant challenges, due to the molecular complexity and
heterogeneity of breast cancer. In this context, the integration of molecular
biomarkers such as microRNAs (miRNAs), small non-coding RNAs with key roles in
post-transcriptional regulation, offers a promising avenue for improving
diagnostic precision and the possibility for personalized treatments.

This dissertation investigates how Machine Learning and Deep Learning
techniques can be used to map miRNA expression profiles to the intrinsic
molecular subtypes of breast cancer: Luminal A, Luminal B, HER2-enriched, and
Basal-like (Triple-Negative). The main research direction is centred on
evaluating the effectiveness of combining miRNA data with clinical variables to
develop robust and interpretable subtype classifiers. Particular attention is
given to comparing discriminative approaches (e.g., Logistic Regression,
XGBoost) with correlation-based or latent space methods (e.g., DIABLO), in
order to identify the most appropriate modelling strategies for multi-omics
integration and evaluate the impact of incorporating data form different
natures.

Preliminary results suggest that ML models can successfully capture miRNA
expression signatures associated with certain subtypes, although Luminal B and
HER2-enriched remain challenging to distinguish. These findings represent an
initial step toward the broader goal of this work: to contribute to the
development of reliable, subtype-specific decision support tools that
incorporate molecular and clinical data for more precise breast cancer
stratification and personalised treatment planning.

% Palavras-chave do resumo em Inglês
% \begin{keywords}
% Keyword 1, Keyword 2, Keyword 3, Keyword 4, Keyword 5, Keyword 6, Keyword 7, Keyword 8, Keyword 9
% \end{keywords}
\keywords{
  microRNAs \and
  breast cancer \and
  biomarkers \and
  machine learning applied to Medicine \and
  multi-omics data
}
